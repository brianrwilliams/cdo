\documentclass[10pt]{amsart}

\usepackage{macros}
\linespread{1.3}

\def\brian{\textcolor{blue}{BW: }\textcolor{blue}}
\def\owen{\textcolor{red}{OG: }\textcolor{red}}


\title{Response to referee re ``Chiral differential operators via\\Batalin-Vilkovisky quantization''}

\begin{document}
\maketitle

We are grateful for the referee's careful examination of our paper, 
which turned up several important issues (now addressed) and caught typos and potentially confusing phrases (also addressed).
The paper is definitely better for his or her feedback.
Thank you!

To facilitate the referee's examination of the revised paper, 
we include here a point by point discussion of what we did.
When the referee looks at the new version, 
he or she will also notice that we have marked places where we made edits:
on the right margin is a blue phrase "Edits here."
We hope this will also make it easier to see what's changed.

{\bf Please note that we have slightly changed the title of the paper to more accurately describe the work.
The title is now:  "Chiral differential operators via quantization of the holomorphic $\sigma$-model".}

\subsection*{Referee Report 1}

We've removed any inconsistencies in falsely claiming that the zero modes are associative and act on ${\rm CDO}$. We've also added Remark 3.7 on p. 26. 

\subsection*{Referee Report 2}

1. Fixed

2. Fixed

3. Fixed

4. Fixed - we removed the misleading line and explained the notation

5. Fixed - we added a small remark to explain notation

6. Fixed - we added a small remark to explain our notation and explain our choice (to avoid confusion with fiber products)

7. Fixed

8. Fixed

9. Fixed

10. Fixed

11. Fixed - we adjusted the meaning of the index $Aut_{n,k}$ so that $Aut_{n,1} = GL_n$

12. This was an unfortunate error in definition; what we wrote was wrong and we appreciate having this flagged. We meant a formal coordinate system, in the holomorphic sense. 

13. Fixed

14. Fixed - we clarified where things are happening 

15. Fixed

16. Fixed

17. Modified the definition of $Exp(X)$ to stress that we only require smooth sections. 
Removed later false claims that holomorphic sections exist.

18. We removed the claim about descent to holomorphic vector bundles, because we do not prove the claim and we do not need it in the paper.

19. Fixed - got rid of $\hat{D}$ and added some comment on the meaning of $D$

20. Fixed

21. Fixed

22. We thank the referee for catching incorrect uses of completion (which we fixed throughout Part I). We now match the constructions in the existing literature on CDOs. To Part III we have added a complementary discussion of the subtleties of completion between the factorization and vertex algebras, when we discuss the Vert functor (see Remark 16.9).

23. Fixed

24. Fixed

25. Fixed - "linear" was absurd

26. Fixed

27. We revised this section considerably and added some discussion about the subtleties of base ring and dimension (which we rode over earlier). We hope it's much clearer (and correct!) now.

28. Fixed

29. Fixed

30. Fixed

31. Fixed. See Remark 4.6 and discussion leading up to it.

32. Fixed

33. Fixed

34. Fixed

35. Fixed - added a pointer to Definition 7.1

36. Fixed

37. We made an attempt to add some motivating remarks about the bracket but we feel that the curious but unfamiliar reader ought to turn to our foundational texts for a more serious treatment.

38. Fixed

39. Fixed

40. Fixed

41. Fixed

42. Fixed

43. Fixed

44. Fixed

45. Fixed

46. We have restructured the proof of Proposition 10.4.
Following the referee's helpful suggestion, we cite "Feynman graph integrals and almost modular forms" to explain the subtle analysis involved in regularizing the two vertex wheel.

47. The propagator $P^L_\epsilon$ has noncompact support, but we are only applying this RG flow operator to a free theory here, so support is not affected. (The flow pairs off external legs of the input observable, and there are no interaction terms to connect to.) We do see that what was written is misleading, so we added some clarifying comments. We hope it is better now.

48. The location is fixed by Schwarz's kernel theorem. We added a brief remark to indicate that the location of the integral kernel depends on the domain and range of the operator.

49. Fixed

50. Fixed

51. We use the context of differentiable vector spaces, which provides a precise notion of "varying smoothly" and we address this aspect in the sentence preceding definition 16.2. If the referee feels we should nonetheless elaborate, we can.


\end{document}