\documentclass[10pt]{amsart}

\usepackage{macros}
\linespread{1.3}

\def\brian{\textcolor{blue}{BW: }\textcolor{blue}}
\def\owen{\textcolor{red}{OG: }\textcolor{red}}


\title{Response to referee re ``Chiral differential operators via\\Batalin-Vilkovisky quantization''}

\begin{document}
\maketitle

We are grateful for the referee's careful examination of our paper, which turned up several important issues (now addressed) and caught typos and potentially confusing phrases (also addressed).
The paper is definitely better for his or her feedback.
Thank you!

To facilitate the referee's examination of the revised paper, 
we include here a point by point discussion of what we did.

1. Fixed

2. Fixed

3. Fixed

4. Fixed - we removed the misleading line and explained the notation

5. Fixed - we added a small remark to explain notation

6. Fixed - we added a small remark to explain our notation and explain our choice (to avoid confusion with fiber products)

7. Fixed

8. Fixed

9. Fixed

10. Fixed

11. Fixed - we adjusted the meaning of the index $Aut_{n,k}$ so that $Aut_{n,1} = GL_n$

12. This was an unfortunate error in definition; what we wrote was wrong and we appreciate having this flagged. We meant a formal coordinate system, in the holomorphic sense. 

13. Fixed

14. \owen{NOT fixed}

15.

16.

17. Modified the definition of $Exp(X)$ to stress that we only require smooth sections. 
Removed future false claims that holomorphic sections exist.

18. We removed the claim about descent to holomorphic vector bundles, because we do not prove the claim and we do not need it in the paper.

19. Fixed - got rid of $\hat{D}$ and added some comment on the meaning of $D$

20. 

21.

22.

23. FIxed

24. Fixed

25. Fixed - "linear" was absurd

26. Fixed

27.

28. 

29. Fixed

30. Fixed

31. 

32. Fixed

33. Fixed

34. Fixed

35. Fixed - added a pointer to Definition 7.1

36. Fixed

37.

38.

39.

40. Fixed

41.

42.

43. 

44. Fixed

45.

46. We have restructured the proof of Proposition 10.4.
We cite the referee's suggestion "Feynman graph integrals and almost modular forms" to explain the subtle analysis involved in regularizing the two vertex wheel.

47.

48.

49. Fixed

50. Fixed

51.

\end{document}