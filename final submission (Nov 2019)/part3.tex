\part*{Part III: Comparison of the constructions}

\section{Overview}

In this part, we finally relate the two stories we have told: 
we show that the Batalin-Vilkovisky quantization of the curved $\beta\gamma$ system from Part II 
produces the chiral differential operators constructed in Part I. 
The key technical tool is a functor $\Vert$ that extracts a vertex algebra 
from a factorization algebra on $\CC$ satisfying a set of natural conditions.
This tool was introduced in \cite{CG1},
where it was already shown that the formal $\beta\gamma$ system recovers the correct vertex algebra
and an isomorphism was given from $\Vert(\Obs^\q_n)$ to $\hCDO_n$.
But it is more subtle to identify that the BV quantization recovers the correct {\em equivariant} vertex algebra.
To show this, we develop some general arguments that relate factorization algebra derivations with vertex algebra derivations.
From these arguments we swiftly verify that $\Vert(\Obs^\q_n)$ is naturally isomorphic to $\hCDO_n$ 
as a $(\TVect,\GL_n)$-equivariant vertex algebra.

Thanks to the machinery of Gelfand-Kazhdan descent, we then deduce our main result.

\begin{thm} 
Let $X$ be a complex $n$-manifold together with a trivialization $\alpha$ of $\ch_2(T_X) \in H^2(X ; \Omega^{2,hol}_{cl})$. Then the factorization algebra $\Obs^{\q}_{X,\alpha}$ obtained by Gelfand-Kazhdan descent 
determines a sheaf of vertex algebras $\Vert(\Obs^{\q}_{X,\alpha})$ on $X$. 
Moreover, there is an isomorphism of sheaves of vertex algebras on $X$
\ben
\Phi : \CDO_{X,\alpha} \xto{\cong} \Vert(\Obs^{\q}_{X,\alpha})
\een
that is natural in the choice of trivialization~$\alpha$.
\end{thm}

Another goal of this paper is to show how physical arguments about the curved $\beta\gamma$ system
are transformed into vertex algebra arguments.
Thus, as a short coda, we review the treatments by Witten \cite{WittenCDO} and Nekrasov \cite{Nek}, 
and we indicate how their approaches are related to our methods.

\section{From factorization to vertex algebras}

Our central challenge now is to relate the vertex algebra produced in Part I with the factorization algebra produced in Part II.
Although factorization algebras are more flexible and general than vertex algebras
--- appearing in every dimension, for instance, and not just on Riemann surfaces --- 
there are recognition criteria that guarantee when a factorization algebra on $\CC$ recovers a vertex algebra.
In essence, the vector space of the vertex algebra is determined by the value of the factorization algebra on a disk,
and the vertex operators are determined by the structure map for two disjoint disks sitting inside a larger disk (i.e., by a flattened pair of pants).
Chapter 5 of \cite{CG1} is devoted to a careful treatment of this relationship 
and constructs a functor $\Vert$ from a certain category of ``holomorphic'' factorization algebras on $\CC$ to the category of vertex algebras.
This chapter also includes a detailed examination of the free $\beta\gamma$ system and its associated vertex algebra.
Here we will overview the main theorem relating factorization and vertex algebras, 
which requires us to introduce some terminology and machinery we need for our main goal.

Two kinds of technical issues appear in formulating the theorem:
\begin{itemize}
\item describing how the structure maps can ``vary holomorphically'' and
\item pinning down various functional analytic aspects.
\end{itemize}
The first involves ideas essential to the goal of this paper, so we dwell a bit on it.
The second is resolved essentially automatically, given our context and the results from \cite{CG1},
but we discuss it briefly.

\subsection{Translation and derivations}

We need to be able to talk about the structure maps in families in order to say that they vary holomorphically.
Our earlier definition of factorization algebras, however, works with the collection of opens in $\CC$ as a set, 
with no topological---much less complex-analytic---structure.
It is straightforward to introduce variations of the definitions with such structure and 
that manifestly contain the examples we've constructed here.

\begin{dfn}
For $U \subset \CC$ and $z \in \CC$, let 
\[
\tau_z U = \{ w \in \CC \,:\, w - z \in U \}
\]
denote the translation of $U$ by $z$. 
Then a factorization algebra $\cF$ on $\CC$ is (discretely) {\em translation-invariant} if we have an isomorphism
$\tau_z: \cF(U) \cong \cF(\tau_z U)$ for every open $U$ and every $z \in \CC$ satisfying
\begin{enumerate}
\item[(i)] for any $z, z'$, $\tau_{z} \circ \tau_{z'} = \tau_{z+z'}$ and
\item[(ii)] for any disjoint open subsets $U_1,\dots, U_k$ in $V$, the diagram
\[
\xymatrix{
\cF(U_1) \otimes \dots \otimes \cF(U_k)  \ar[r]^-{\tau_z} \ar[d] &
\cF(\tau_z U_1) \otimes \dots \otimes \cF(\tau_z U_k)  \ar[d] \\
\cF(V) \ar[r]^{\tau_x} & \cF(\tau_z V)
}
\]
commutes. (Here the vertical arrows are the structure maps of the factorization algebra.) 
\end{enumerate}
\end{dfn}

Note that the sheaf of holomorphic functions $\cO$ on $\CC$ satisfies the sheaf-theoretic version of this definition,
as does the Dolbeault complex. 
In consequence, the factorization algebras $\Obs^\cl_n$ and $\Obs^\q_n$ are translation-invariant.

We now turn to talking about families.
Let $\cl(U)$ denote the closure of an open set $U \subset \CC$.
Given $U_1,\ldots,U_n$ disjoint opens in $V$, let
\[
{\rm Conf}(U_1,\ldots,U_n \,|\, V) = \{ (z_1,\ldots,z_n) \in \CC^n \,:\,\forall i \neq j ,\, \cl(\tau_{z_i} U_i) \cap  \cl(\tau_{z_j} U_j) =\emptyset \text{ and }\forall i,\, \cl(\tau_{z_i} U_i) \subset V \}.
\]
In other words, this open subset of $\CC^n$ parametrizes all the translations of the $U_i$ that keep them in $V$ 
and keep their closures disjoint. (It will suffice to focus on collections $U_i$ whose closures are disjoint.)
This space ${\rm Conf}(U_1,\ldots,U_n \,|\, V)$ inherits the structure of a complex manifold from $\CC^n$. 

Now let $\cF$ be a discretely translation-invariant factorization algebra. 
We can use the isomorphisms to replace any appearance of $\cF(\tau_z U)$ with $\cF(U)$.
Hence for each point $(z_1,\ldots,z_n) \in {\rm Conf}(U_1,\ldots,U_n \,|\, V)$, we have a structure map
\[
m_{(z_1,\ldots,z_n)}: \cF(U_1) \otimes \cdots \otimes \cF(U_n) \to \cF(V)
\]
by the composite
\[
\cF(U_1) \otimes \cdots \otimes \cF(U_n) \to \cF(\tau_{z_1} U_1) \otimes \cdots \otimes \cF(\tau_{z_n} U_n) \to \cF(V),
\]
where the first map is the tensor product of translation maps $\tau_{z_i}$ and the second map is the structure map of $\cF$.

To talk about these structure maps varying smoothly over ${\rm Conf}(U_1,\ldots,U_n \,|\, V)$, 
we need the factorization algebra to take values in vector spaces (or cochain complexes thereof) 
in which one can talk about smooth families.
We will work in the context described in Section \ref{sec functional analysis}, 
where the topology (or bornology) provides a precise notion of smooth families of linear maps.

\begin{dfn}
A translation-invariant factorization algebra $\cF$ on $\CC$ is {\em smoothly} translation-invariant 
if: 
\begin{enumerate}
\item[(i)] For any collection of opens $U_1,\ldots,U_n$ in $V$ whose closures are pairwise disjoint, 
the maps $m_{z_1,\dots,z_n}$ depend smoothly on $(z_1,\dots,z_k) \in {\rm Conf}(U_1,\ldots,U_n \,|\, V)$.
\item[(ii)] The factorization algebra $\cF$ is equipped with an action by derivations of the abelian Lie algebra $\RR^2$ of translations. 
For $v \in \RR^2$ and open $U \subset \CC$, we will denote the corresponding derivation by $\d/\d v : \cF(U) \to \cF(U)$.
This Lie algebra action is viewed as an infinitesimal version of the global translation invariance.
\item[(iii)] This infinitesimal action is compatible with the global translation invariance in the following sense. 
For $v \in \RR^2$, let $v_i \in (\RR^2)^n$ denote the vector $(0,\ldots,v,\ldots,0)$, with $v$ placed in the $i$-slot and $0$ in the other $n-1$ slots.  If $\alpha_i \in \cF(U_i)$, then we require that
$$
\frac{\d}{\d v_i} m_{z_1,\dots,z_n} (\alpha_1,\dots,\alpha_n) = m_{z_1,\dots,z_n}\left(\alpha_1,\dots,\frac{\d}{\d v} \alpha_i, \dots, \alpha_n\right).
$$
\end{enumerate}  
\end{dfn}

The translation Lie algebra is real. 
As $\cF$ is defined over $\CC$, we can extend the action to the complexified translation Lie algebra $\RR^{2} \otimes_{\RR} \CC$.  
We will denote by $\partial_z$ and $\partial_{\zbar}$ the derivations on $\cF$ associated to the obvious vector fields on $\CC$.
To be {\em holomorphic}, we want the vector field $\partial_{\zbar}$ to act homotopically trivially on $\cF$.

\begin{dfn}\label{holomorphically_translation_invariant_definition}
A translation-invariant prefactorization algebra $\cF$ on $\CC$ is \emph{holomorphically translation-invariant} 
if it is equipped with a derivation $\eta: \cF \to \cF$ of cohomological degree $-1$ such that
\[ 
\d \eta = \partial_{\zbar}, \; [\eta,\eta] = 0,\,\text{and}\,  \left[ \eta,  \partial_{\zbar}  \right] = 0. 
\]
Here $\d$ refers to the differential on the dg Lie algebra ${\rm Der}(\cF)$.
\end{dfn}

\subsection{Rotation and decomposition}

In practice, we are interested in $\cF$ where the action by translation extends to an action of orientation-preserving Euclidean transformations of $\CC$.

\begin{dfn}
A \emph{holomorphically translation-invariant prefactorization algebra $\cF$ on $\CC$ with a compatible $U(1)$ action} 
is a smoothly $U(1) \ltimes \RR^2$-invariant prefactorization algebra $\cF$ 
together with an extension of the action of the complex Lie algebra 
$$
{\rm Lie}_{\CC}(U(1) \ltimes \RR^2) = \CC\left\{ \partial_\theta, \partial_z, \partial_{\zbar} \right\},
$$
where $\partial_\theta$ is a basis of ${\rm Lie}_\CC(U(1))$, to an action of the dg Lie algebra
$$
\CC\left\{ \partial_\theta, \partial_z, \partial_{\zbar} \right\} \oplus \CC\{\eta\} ,
$$
where $\eta$ is of cohomological degree $-1$ and the differential is $\d \eta = \partial_{\zbar}$.
In this dg Lie algebra, all commutators involving $\eta$ vanish except for $[\partial_\theta, \eta] =~-\eta$.
\end{dfn}

The theorem on vertex algebras requires a technical hypothesis regarding the $U(1)$-action on the factorization algebra $\cF$:
we need this action to be {\em tame}, in the following sense. 

For any compact Lie group $G$, the space $\mathcal{D}(G)$ of distributions on $G$ is an algebra under convolution.  
The convolution product $\ast$ is smooth in the sense that it varies nicely in families, as per our approach to functional analysis,
so that the algebra structure is smooth in families.
There is a natural map $\delta: G \to \mathcal{D}(G)$
sending an element $g$ to the $\delta$-function at $g$.  
It is a smooth map and a homomorphism of monoids.  

\begin{dfn} 
A \emph{tame action} of $G$ on a vector space $V$ of the type discussed in Section \ref{sec functional analysis} (e.g., convenient) is a smooth action of the algebra $\mathcal{D}(G)$ on $V$.  
(Note that this means $G$ acts on $V$ via composition $G \to \mathcal{D}(G)^{\times}$ sending $g$ to $\delta_g$.)
For $V$ a cochain complex of such vector spaces, a tame action commutes with the differential on $E$. 
\end{dfn}

The case $G = U(1)$ is the only one relevant for us here. 
For each integer $k$, the function $\rho_k: e^{i\theta} \mapsto e^{ik\theta}$ encodes an irreducible representation of $U(1)$.
It determines a distribution $\rho_k \, \d \theta$ on $U(1)$ that we will abusively call $\rho_k$ as well.
In $\mathcal{D}(U(1))$, the element $\rho_k$  is an idempotent.

\begin{dfn}
Let $V$ be equipped with a tame action of $U(1)$, which we will denote by $\ast$. 
Let $V_k$ denote the {\em weight $k$ eigenspace} for the $U(1)$-action on $V$.
The map $\rho_k \ast - : V \to V$ defines a projection from $V$ onto~$V_k$. 
\end{dfn}

\subsection{The theorem about $\Vert$}

We now turn to the main theorem from Chapter 5 of \cite{CG1}, 
which provides a functor from a certain category of factorization algebras on $\CC$ to the category of vertex algebras.

\begin{dfn}
Let $\cF$ be a tamely $U(1)$-equivariant holomorphically translation-invariant factorization algebra on $\CC$.
Let $\cF_k(D_r(0))$ denote the subcomplex of weight $k$ eigenspaces in $\cF(D_r(0))$, 
the value of $\cF$ on a radius $r$ disk around the origin.
Then $\cF$ is {\em amenably holomorphic} if it satisfies the following conditions:
\begin{enumerate}
\item 
For every pair of radii $r < r'$, the structure map
$$
\cF_k(D(0,r)) \to \cF_k(D(0,r'))
$$
is a quasi-isomorphism.
\item For $k \gg 0$, the vector space $H^\ast (\cF_k(D(0,r))$ is zero.  
\item For each $k$ and $r$, we require that $H^\ast(\cF_k(D(0,r) )$ is isomorphic to a countable sequential colimit of finite-dimensional graded vector spaces. 
\end{enumerate}
\end{dfn}

Observe that for an amenably holomorphic factorization algebra $\cF$, 
the vector space $H^* (\cF_k(D(0,r))$ is independent of $r$ by assumption.
Let 
\[
V_\cF= \bigoplus_{k \in \ZZ} H^* (\cF_k(D(0,r))
\]
be the direct sum of the weight spaces, 
and let 
\[
\overline{V}_\cF = \prod_{k \in \ZZ} H^* (\cF_k(D(0,r))
\]
be the direct product of the weight spaces.
Note that for any disk $D_r(0)$, there is a map $V_\cF \to H^*\cF(D_r(0))$ by the inclusion of the weight spaces.
Likewise, there is map $H^*\cF(D_r(0)) \to \overline{V}_\cF$.
The structure maps of $\cF$ thus determine a family of maps $m_z: V_\cF \otimes V_\cF \to \overline{V}_\cF$ 
given by the composition
\[
V_\cF \otimes V_\cF \to H^*\cF(D_r(0)) \otimes H^* \cF(D_r(z)) \to \H^*\cF(D_R(0)) \to \overline{V}_\cF,
\]
where the middle map is the structure map of $H^* \cF$ with radii $r$ and $R$ such that $2r < |z|$ and $|z| + r < R$.
By construction, these maps $m_z$ vary holomorphically in the parameter $z~\in~\CC-\{0\}$.

Let $\FA^{am}$ denote the category of amenably holomorphic factorization algebras on $\CC$,
where morphisms are maps of prefactorization algebras intertwining the actions of $\CC$ by translation and $U(1)$ by rotation.
Let $\VA$ denote the category of vertex algebras.

\begin{thm}[Theorem 5.3.3, \cite{CG1}] 
\label{theorem_vertex_algebra}
There is a functor $\Vert:\FA^{am} \to \VA$. 
For $\cF$ amenably holomorphic, the underlying vector space of the vertex algebra is $V_\cF$,
and the vertex operator $Y_{\Vert(\cF)}$ is determined by the maps $m_z$ arising from the structure maps of~$\cF$.
\end{thm}

\begin{rmk}
\label{rmk on completions}
As remarked in \cite{CG1} immediately following the theorem,  
this construction makes sense on a factorization algebra $\cF$ that is the inverse limit $\lim_k \cF_k$ of amenably holomorphic factorization algebras,
so that $\Vert(\cF)$ is the inverse limit $\lim_k \Vert(\cF_k)$ of the associated vertex algebras.
This variant is needed by use to recover the completed CDOs~$\hCDO_n$ from the $\beta\gamma$ observables where the target is the formal $n$-disk.
\end{rmk}

Immediate consequences of this theorem include the following.

\begin{lemma}\label{factder}
Let $X \in \Der(\cF)$ be a derivation of $\cF$ as an amenably holomorphic factorization algebra. In particular, we require that $X$ commutes with translation, $[X, \partial_z] = 0$, $[X, \partial_{\zbar}] = 0$, and $[X, \partial_\theta] = 0$. Then $X$ induces a vertex algebra derivation $V_X$ on $\Vert(\cF)$.
\end{lemma}

We call such derivations {\em amenably holomorphic}.

\begin{proof}
Let $D_r$ denote the disk of radius $r$ centered at the origin. 
Let $\cF_k(D_r)$ denote the weight $k$ subspace of $\cF(D_r)$ with respect to the rotation action of $S^1$. 
By hypothesis, $X$ preserves the weight spaces: $\partial_\theta$ acts on $\cF_k(D_r)$ by multiplication by $k$, 
and since $X$ commutes with $\partial_\theta$, it preserves each weight space. 
Hence $X$ induces a linear map $V_X$ on $V_\cF = \bigoplus_k H^*(\cF_k(D_r))$. 

As $X$ is a derivation, it intertwines with the structure maps of $\cF$. In particular, for any two small disjoint disks $D_1$ and $D_2$ included into a larger disk $D_{big}$, we see that 
\[
X_{D_{big}}(m^{D_1,D_2}_{D_{big}}(v_1, v_2)) = m^{D_1,D_2}_{D_{big}}(X_{D_1} v_1, v_2) \pm m^{D_1,D_2}_{D_{big}}(v_1, X_{D_2} v_2).
\]
Since the action of $X$ equivariant with respect to affine transformations of $\CC$, we see that this derivation property holds for the one-parameter family of ``multiplication" operations 
\[
m_{z,0}: V_\cF \otimes V_\cF \to \overline{V}_\cF[[z,z^{-1}]].
\]
Hence the action of $V_X$ on $V_\cF$ is a derivation of the vertex operator map $Y$, which is given by the Laurent expansion of $m_{z,0}$. 
\end{proof}

By a similar but easier argument, we find the following.

\begin{lemma}
Let $\phi: \cF \to \cF$ be an automorphism of an amenably holomorphical factorization algebra $\cF$. Then $\phi$ induces an automorphism of vertex algebras $V_\phi: V_\cF \xto{\cong} V_\cF$.
\end{lemma}

Note that in both these situations we require that the derivation or
automorphism commutes on the nose with all the equivariant structure on $\cF$. 
Such a strict situation is adequate for our purposes here. 
(The homotopical versions of these statements should hold but will not be pursued.)

We now wish to apply these lemmas to the case where the factorization algebra has an action of a pair $(\fg,K)$. 
The data of a semi-strict $(\fg,K)$-structure on $\cF$ involves a group homomorphism $\rho^K : K \to \Aut(\cF)$
together with an $L_\infty$-homomorphism $\rho^\fg : \fg \to \Der(\cF)$
that satisfy the compatibilities in Definition~\ref{dfn ss HC mod}. 

\begin{cor}\label{vertpair} 
Let $\cF$ be an amenably holomorphic factorization algebra 
together with the structure of a semi-strict $(\fg,K)$-module. where
$\fg$ acts by derivations and $K$ acts by automorphisms of the amenably holomorphic factorization
algebra. 
Then the $\ZZ_{\geq 0}$-graded vertex algebra $\Vert(\cF)$ has a
strict action of the pair $(\fg,K)$. 
\end{cor}

\section{Observables for the formal $\beta\gamma$ system}

As an example of the relationship encoded in $\Vert$,
the free $\beta\gamma$ system is examined in depth in Chapter 5 of \cite{CG1}.
It is shown there that the factorization algebra $\Obs^{\fr,\q}_n$ of quantum observables for the free $\beta\gamma$ system is amenably holomorphic,
and it is also shown that the associated vertex algebra is precisely the usual $\beta\gamma$ vertex algebra.

\begin{thm}[Theorem 5.3.3.2, \cite{CG1}]
The factorization algebra $\Obs^{\q,\fr}_n$ is amenably holomorphic. 
Moreover, 
there is an isomorphism of $\ZZ_{\geq 0}$-graded vertex algebras 
\[
\Phi^{\fr}_n : \CDO_n \xto{\cong} \Vert(\Obs^{\q,\fr}_n)|_{\hbar= 2 \pi i}
\] 
after specializing $\hbar=2 \pi i$.
The map is $\GL_n$-equivariant.
\end{thm}

As noted in Remark~\ref{rmk on completions}, 
these results immediately imply the analogous results about the completed case, 
where the free $\beta\gamma$ system is replaced by having the target be the formal $n$-disk.
Hence, in combination with Proposition~\ref{equiv of noneq},
we obtain the following result.

\begin{cor}
The factorization algebra $\Obs^\q_n$ is amenably holomorphic. 
Moreover, there is a $\GL_n$-equivariant isomorphism of $\ZZ_{\geq 0}$-graded vertex algebras 
$\Phi_n : \hCDO_n \xto{\cong} \Vert(\Obs^\q_n)|_{\hbar= 2 \pi i}$.
\end{cor}

\begin{rmk}
The reader might (correctly) object that $\Obs^\q_n$ lives in modules over $\CC[[\hbar]]$,  
and so one cannot specialize $\hbar$ to a nonzero value.
We note, however, that the differential on $\Obs^\q_n$ has the form $D_0 + \hbar D_1$, 
so that only a single power of $\hbar$ appears.
Hence the construction of the quantum observables is well-defined in modules over $\CC[\hbar]$,
where one can specialize $\hbar$ to a nonzero value,
and we are using that version of $\Obs^\q_n$ here.
\end{rmk}

Section \ref{sec loc sym} tackles the much more subtle challenge of showing the isomorphism $\Phi_n$ is $\TVect$-equivariant,
and hence that we get an isomorphism of $(\TVect,\GL_n)$-equivariant vertex algebras.
This property is crucial for applying Gelfand-Kazhdan descent 
and hence recognizing that the BV quantization truly does recover chiral differential operators.

In the remainder of this section, we review some aspects of the $\beta\gamma$ system's factorization
and vertex algebras that are useful for our central goal.

\begin{rmk}
An extensive and expository treatment of these aspects appears in Chapter 6 of~\cite{GwThesis}.
\end{rmk}

\subsection{Some useful identifications}
\label{identifications}

It will be useful to understand explicitly how to identify a representative in the factorization algebra for an element in the vertex algebra.
To be more precise, the construction $\Vert$ ensures that given $v \in \hCDO_n$, there is some cohomology class $[O_v]$ in $H^* \Obs^\q_n(D_r(0)$.
We would like to have a {\em cochain} representative $O_v$ in the disk observables $\Obs^\q_n(D_r(0))$ as well.
Similarly, given a Fourier mode $v_{(n)}$, 
we would like to know a cochain representative $O_{v_{(n)}}$ in the annular observables $\Obs^\q_n(A_{r<R}(0))$.
Although the functor $\Vert$ ensures these wishes can be fulfilled,
the formulas may be quite complicated.

We now examine this issue, starting with the classical observables, where the situation is simpler,
before turning to the quantum observables.
To minimize the number of indices, we restrict to $n = 1$; 
hence, we have elements $b_n$ and $c_m$ with no upper index. 
The extension to arbitrary $n$ is straightforward:
just reinsert the superscripts, e.g., use $c^j_m$ and not just $c_m$.

For the classical observables and $\Gr\, \hCDO_n$,
Cauchy's formula provides explicit integral expressions for the most important linear observables
(i.e., distributions on the fields $\gamma$ and $\beta$).
For example, set 
\[
O_{c_{-m}}(\gamma,\beta) = \frac{m!}{2 \pi i} \int_{|z| = 1} \frac{\gamma(z)}{z^{m+1}} \d z.
\]
This linear observable simply reads off the coefficient of $z^m$ in the power series expansion of a holomorphic $\gamma$.
The support of this distribution is the unit circle, 
so that we can view $O_{c_n}$ as a cocycle in $\Obs^{\cl}_1(A_{r<R}(0))$ for any annulus with $r < 1 < R$.
But it also provides a cocycle in $\Obs^{\cl}_1(D_R(0))$, 
and this cocycle is a representative of the element $c_n$ in $\Gr\, \hCDO_1$.
Similarly, a cochain representative of $b_n$ is
\[
O_{b_{-l}}(\gamma,\beta) = \frac{(l-1)!}{2 \pi i} \int_{|z| = 1} \frac{\beta(z)}{z^{l}},
\]
which reads off the coefficient of $z^{l+1}$ in the power series expansion of a holomorphic one-form~$\beta$.

It is thus easy to provide explicit representatives for monomials like $b_{i_1} \cdots b_{i_l} c_{j_1} \cdots c_{j_m}$.
One simply takes the obvious product --- in the symmetric algebra of distributions --- of the representatives just given.

It is also straightforward to produce smeared versions of these observables, if one wants 
(and we will want it shortly).
Fix a bump function $f(r)$ on some interval $(a,b)$, with $0 < a$, 
such that $\int_a^b f(r) \, \d r = 1$.
Then
\[
O'_{c_{-m}}(\gamma,\beta) = \frac{m!}{2 \pi i}\int_a^b \int_{|z| = r} \frac{\gamma(z)}{z^{m+1}} \d z \, f(r) \d r
\]
is a smeared representative of $c_m$.

Note that if one is working with an annulus rather than disk, then negative powers of $n$ are allowed in the denominator.
In this setting the cocycles $O_{c_m}$ and $O_{b_l}$ read off Laurent coefficients.
As observables on an annulus, they correspond to Fourier modes from the point of view of vertex algebras.
To be explicit, the zeroth Fourier mode $(c_m)_{(0)}$ is represented by $O_{c_m}$ viewed as an observable on an annulus.
Let us explain why.

The vertex operator on $\Gr\,\CDO$ admits a concrete interpretation in terms of ``observing'' coefficients of expansions.
The element $Y(c_m;w)$ should be viewed as an observable on the annulus:
given $\gamma$ a holomorphic function on the annulus and $w$ a point on that annulus, 
$Y(c_m;w)$ measures the coefficient of $(z-w)^m$ in the power series expansion of $\gamma$ around $w$.
If we know the Laurent expansion of $\gamma$ around $0$, then we can provide an expression for this coefficient.

For instance, if we know a Laurent expansion
\[
\gamma(w) = \sum_{m \in \ZZ} c_{-m} w^m,
\]
then
\[
Y(c_0;w)(\gamma) = \sum_m w^m c_{-m}(\gamma).
\]
In the first line, we view the $c_m$ as numbers, providing the Laurent coefficients of $\gamma$,
but in the second line, we view the $c_m$ as operators, providing the Laurent coefficients of~$\gamma$.
In consequence, we see that the $m$th Fourier mode $(c_0)_{(-m)}$ has 
\[
O_{c_{-m}}(\gamma,\beta) = \frac{m!}{2 \pi i} \int_{|z| = 1} \frac{\gamma(z)}{z^{m+1}} \d z
\]
as an explicit representative.

From the factorization algebra point of view, 
the vertex operator amounts to saying that the observable
\[
O_{c_0,w}(\gamma,\beta) = \frac{1}{2 \pi i} \int_{|z-w| = \epsilon} \frac{\gamma(z)}{z-w} \d z,
\]
which measures the value of $\gamma$ at $w$, is cohomologous to the observable
\[
\sum_{m \in \ZZ} w^m O_{c_{-m}}(\gamma,\beta)
\]
in $\Obs^{\cl}(A_{r<R}(0))$ with for $r < |w| < R$ and $\epsilon$ sufficiently small. 

We now turn to providing a tool for understanding how the quantum observables
and $\hCDO_n$ relate.

\subsection{Quantizing observables}
\label{sec quant map}

For the free $\beta\gamma$ system on $\CC$, 
there is a natural cochain isomorphism 
\[
\frak{q}_U: \Obs^{\cl,\fr}(U)[\hbar] \xto{\cong} \Obs^{\q,\fr}(U)
\]
between the classical and quantum observables on any fixed open $U$.
(Recall that the superscript $\fr$ means the smooth or smeared observables.
See Section \ref{noneqsec}.)
This isomorphism ``promotes'' a classical observable to a quantum observable.
It does not preserve, however, the structure maps of the factorization algebras,
and so we view the quantum observables as deforming the structure maps of $\Obs^{\cl,\fr}$ in an interesting, $\hbar$-dependent way:
for $U, U'$ disjoint opens in $V$, the ``quantized'' structure map sends observables 
$F \in \Obs^{\cl,\fr}(U)$ and $F' \in \Obs^{\cl,\fr}(U')$ to 
\[
F \star F' = \frak{q}_{V}^{-1}(\frak{q}_U(F) \cdot \frak{q}_{U'}(F')) \in \Obs^{\cl,\fr}(V)[\hbar],
\]
where $\cdot$ denotes the factorization product in $\Obs^{\q,\fr}$.
We use $\star$ to emphasize that we are ``deformation-quantizing'' the factorization product on the classical observables.

This description allows one to understand concretely how BV quantization affects the factorization algebra,
since the classical observables $\Obs^{\cl,\fr}(U)$ are very explicit and simply amount to algebraic functions on the space of holomorphic functions $\cO(U)$ and holomorphic one-forms~$\Omega^1_{hol}(U)$.
Details of this construction can be found in Section 6, Chapter 4 and Section 3, Chapter 5 of~\cite{CG1}.

To construct the map $\frak{q}$, we use the fact that on $\CC$, 
the operator $\dbar$ possesses a natural choice of propagator (or Green's function),
namely
\[
P(z,w) = \frac{1}{2\pi i}\frac{\d z + \d w}{z - w}.
\]
This distributional one-form on $\CC^2$ satisfies $(\dbar \otimes 1) P = \delta_{\Delta}$, 
where $\delta_{\Delta}$ is the delta-current supported along the diagonal and providing the integral kernel for the identity.
One can view this one-form as a distributional section of the fields $\gamma$ and $\beta$:
for example, for fixed $w$, the one-form $\d z/(z-w)$ is a $\beta$ field in the $z$-variable as it is a $(1,0)$-form,
and dually for the other term in~$P$.

This element $P$ also defines a second-order differential operator $\partial_P$ on the commutative algebra $\Obs^{\cl,\fr}(U)$.
Let us recall the general algebraic context.
For any symmetric algebra $\Sym(V^*)$, an element $v \in V$ defines a vector field $\partial_v$ via contraction:
given $f \in \Sym^{n+1}(V^*)$, we set 
\[
\partial_v f(x_1 \otimes \cdots \otimes x_n) = f(v \otimes x_1 \otimes \cdots \otimes x_n),
\]
by viewing $f$ as an $S_n$-invariant element of $(V^*)^{\otimes n}$.
Similarly, given $p \in V^{\otimes 2}$, we define a second-order differential operator $\partial_p$ by contraction.
Recall that the classical observables $\Obs^{\cl,\fr}(U)$ are a symmetric algebra,
and let $\partial_P$ be the operator obtained by contraction.

\begin{dfn}
Define the {\em promotion map}
\[
\begin{array}{cccc}
\frak{q}:& \Obs^{\cl,\fr}(U)[\hbar] &\to& \Obs^{\q,\fr}(U)\\
& F & \mapsto & \exp(\hbar \partial_P)F
\end{array}.
\]
In other words, one applies a version of Wick contraction to any classical observable $F$,
repeatedly contracting away two inputs with the propagator.
\end{dfn}

\begin{rmk}
In terms of the RG flow used in Part II, this map $\frak{q}$ encodes flowing to length scale $L = \infty$.
Because our theory is free and we restrict to smeared observables, this operation is well-defined.
Effectively, it describes the relations between observables after integrating out the nonzero modes.
\end{rmk}

\subsection{An example}
\label{circ ex}

Consider the classical observable on the annulus $A = \{1/2 < |z| < 3/2\}$ given by
\[
F(\gamma,\beta) = \frac{1}{2\pi i}\int_{|z| = 1} \gamma \wedge \beta,
\]
for $\gamma \in C^\infty(A)$ and $\beta \in \Omega^{1,0}(A)$.
(We say $F$ vanishes if $\gamma$ is a $(0,1)$-form or if $\beta$ is a $(1,1)$-form.)
Its cohomology class $[F]$ in $H^0 \Obs^{\cl,\fr}(A)$ encodes a function on $\cO(A)$ and $\Omega^1_{hol}(A)$
where for 
\[
\gamma = \sum_{n \in \ZZ} c_{-n} z^n \quad \text{and} \quad \beta = \sum_{n \in \ZZ} b_{-n} z^n \, \d z,
\]
we have
\[
[F](\gamma,\beta) =  \sum_{-m - n = -1} c_m b_n
\]
by Cauchy's integral formula.
As $F$ is not a smeared classical observable, 
we cannot immediately apply $\frak{q}$ but first must replace it by a cohomologous smeared observable $\widetilde{F}$.
(If one tries to evaluate $\partial_P F$, one finds it is ill-defined.)

Here is one approach to smearing among many.
Note that the functional
\[
H(\gamma,\beta) = \frac{1}{(2\pi i)^2} \int_{|z| = r} \int_{|w| = R} \frac{\gamma(z) \d z \wedge \beta(w)}{z-w},
\]
with $R > r$, is cohomologous to $F$.
(Simply plug in holomorphic $\gamma$ and $\beta$ and use Cauchy's theorems.)
This functional $H$, while distributional, is easier to ``smear''
by letting $r$ and $R$ vary.
Fix a compactly supported bump function $f(r,R)$ on $B = (1/2,1) \times (1,3/2)$ such that $\int_B f(r,R) \,\d r \, \d R = 1$.
Define
\[
\widetilde{F}(\gamma,\beta) = 
\frac{1}{(2\pi i)^2} \int_B  \int_{|z| = r} \int_{|w| = R} \frac{\gamma(z) \d z \wedge \beta(w)}{z-w}\, f(r,R) \,\d r \, \d R.
\]
Then $\frak{q}(\widetilde{F}) = \widetilde{F}$, since 
\[
\partial_P \widetilde{F} = \widetilde{F}(P(z,w)) = \frac{1}{(2\pi i)^2} \int_B  \int_{|z| = r} \int_{|w| = R} \frac{\d z \wedge \d w}{(z-w)^2}\, f(r,R) \,\d r \, \d R = 0.
\]
In fact, the smearing was not necessary here: the contraction $\partial_P H$ is already well-defined.

\begin{rmk}
\label{promotion}
This approach works well for classical observables with simple descriptions, like our $F$ above.
The initial formula might involve integrating some polynomial in $\gamma$ and $\beta$ around a circle,
but one can replace it, up to cohomology, by an integral over a collection of disjoint circles,
where each copy of $\gamma$ and $\beta$ has its own circle.
Our $H$ is constructed in such a fashion.
Once the supports of these circles are disjoint, 
one can apply $\frak{q}$ directly, without smearing.
\end{rmk}

\section{Local symmetries acting on observables}
\label{sec loc sym}

Our goal here is to articulate how local symmetries of a field theory like the $\beta\gamma$ system
produce derivations of the associated vertex algebras. 
The core construction makes sense for any BV theory 
but we will focus on a version applicable here.
(These manipulations are certainly well-known in the physics literature; 
our work just articulates them in the language of factorization algebras.)

\subsection{General arguments}

Every local functional $L$ in a field theory provides both a derivation of the observables 
and an observable itself.
We want to understand how these two manifestations of $L$ relate.

The derivations arise as ``Hamiltonian vector fields.''
Consider the map of dg Lie algebras
\[
\begin{array}{cccc}
{\rm Ham}: & \cO_{loc}[-1] & \to & \Der(\Obs_T^\cl) \\
& L & \mapsto & \{L,-\} 
\end{array}.
\]
(See Section 3, Chapter 5 of \cite{CosBook} for a discussion of this construction.)
In other words, a local functional can be viewed as a symmetry of the classical field theory.
Note that this map naturally extends to a map of graded Lie algebras into $\Der(\Obs_T^q)$,
but it does not intertwine the differentials,
which is an example of why classical symmetries might not quantize.

We would like to view some of these symmetries as ``inner,''
i.e., realized as the factorization product with an observable,
just as an inner derivation of an associative algebra means it is given by bracketing with an element of the algebra.
To compare derivations to factorization products, however, 
we need to be able to turn local functionals into observables.
A minor issue is that local functionals need not have compact support
and hence do not provide observables on fields with non-compact support.
This problem is easy to fix.

Let $L$ be a local functional,
and let $\cL$ denote the Lagrangian density such that $L = \int \cL$.
By this we mean that if $\gamma$ and $\beta$ are fields with compact support, 
then
\[
L(\gamma,\beta) = \int_\CC \cL(\gamma,\beta).
\]
Let $K \subset \CC$ be a compact subset whose boundary $\partial K$ is a smooth submanifold.
Set $L_K = \int_K \cL,$ so that one simply integrates over $K$ rather than all of $\CC$.
As $K$ is compact, we see that $L_K$ is a well-defined observable on all fields, not just those with compact support.

In short, for $U$ an open set containing the compact submanifold $K$, we have a cochain map
\[
\begin{array}{cccc}
(-)_K: & \cO_{loc} & \to & \Obs_n^\cl(U) \\
& L & \mapsto & L_K
\end{array}.
\]
This map extends to quantum observables but no longer respects the differentials.

A direct computation then gives a relationship between the factorization product and the derivation.

\begin{lemma}
Let $F$ be a cocycle in $\Obs_n^\q(U)$ and $K \subset U$ a compact submanifold.
Then
\[
\d^\q(L_K F) = \d^\q(L_K)F + \{ L_K, F\},
\]
where the notation $L_K F$, for instance, denotes the product in the completed symmetric algebra underlying~$\Obs^\q_n$.
Hence, if $\d^\q(L_K)$ has support in $U\setminus K$, then at the level of cohomology
\[
[\{ L_K, F\}] = - [ \d^\q(L_K) \cdot F],
\]
where $\cdot$ denotes the factorization product for the structure map 
$\Obs_T^\q(V \setminus U) \otimes \Obs_T^\q(U) \to \Obs_T^\q(V)$.
\end{lemma}

\begin{rmk}
This relationship between ``local symmetries'' (i.e., given by local functionals {\em aka} local currents) and the operator product of observables is reminiscent of Ward identities.
We will see below an explicit instantiation of this relationship,
but note here one simple consequence of the lemma:
An observable that is killed by $\{L_K,-\}$ classically --- and hence is fixed by that symmetry --- 
may not be killed at the quantum level.
\end{rmk}

We now restrict our attention to local functionals of a special form,
which admit a particularly useful application of this lemma.
(The arguments we develop here apply with minor changes 
to any free BV theory on $\CC$ whose action is given by~$\dbar.$)

Given a finite set of constant-coefficient holomorphic differential operators
\[
D_1,\ldots,D_m \in \CC[\partial/\partial z],
\]
consider the local functional
\[
L(\gamma,\beta) = \int_\CC D_1(\gamma) \wedge \cdots \wedge D_m(\gamma) \wedge \beta.
\]
Note that one must take some care to properly interpret such a functional, 
as with $I^{{\rm W}}$ or $J$ defined in Section \ref{sec obsdef}. 
This functional vanishes except when working over base dg algebras not concentrated in degree zero.

\begin{dfn}
A local functional $L$ is {\em constant-coefficient holomorphic, $\beta$-linear} if it is a sum of local functionals of the form above.
\end{dfn}

A nice property of such a local functional $L$ is that the derivation $\{L,-\}$ is manifestly amenably holomorphic 
because the integrand is translation-invariant and rotation-equivariant. Hence we see the following.

\begin{lemma}
The factorization algebra derivation $\{L,-\}$ induces a derivation $V_L$ on the vertex algebra~$\hCDO_n$.
\end{lemma}

We now wish to find an alternative description of that derivation.
Recall that the differential on $\Obs^\cl_n$ is denoted $\dbar$, 
as it is the extension of $\dbar$ on the linear observables to a derivation on the symmetric algebra.

\begin{lemma}
For a local functional
\[
L(\gamma,\beta) = \int_\CC D_1(\gamma) \wedge \cdots \wedge D_m(\gamma) \wedge \beta,
\]
with $D_1,\ldots,D_m \in \CC[\partial/\partial z]$, there is an equality
\[
(\dbar L_K)(\gamma,\beta) = \int_{\partial K} D_1(\gamma) \wedge \cdots \wedge D_m(\gamma) \wedge \beta,
\]
for any compact submanifold $K$ of~$\CC$.
\end{lemma}

\begin{proof}
This claim follows from Stokes' lemma.
Compute
\begin{align*}
(\dbar L_K)(\gamma,\beta) 
&= \int_K \dbar \left( D_1(\gamma) \wedge \cdots \wedge D_m(\gamma) \wedge \beta. \right)\\
&= \int_K (\d - \partial) \left( D_1(\gamma) \wedge \cdots \wedge D_m(\gamma) \wedge \beta \right)\\
&= \int_K \d \left( D_1(\gamma) \wedge \cdots \wedge D_m(\gamma) \wedge \beta \right)\\
&= \int_{\partial K} D_1(\gamma) \wedge \cdots \wedge D_m(\gamma) \wedge \beta.
\end{align*}
The reason $\partial$ annihilates the integrand is that $\beta$ contributes a $\d z$ term already.
\end{proof}

Consider as well a closely related situation.

\begin{dfn}
A local functional $L$ is {\em constant-coefficient holomorphic, $\beta$-free} if it is a sum of local functionals of the form
\[
\int_\CC D_1(\gamma) \wedge \cdots \wedge D_m(\gamma) \wedge \d z
\]
where the $D_j$ are constant-coefficient holomorphic differential operators.
\end{dfn}

\begin{lemma}
For a constant-coefficient holomorphic, $\beta$-free local functional $L = \int \cL$,
there is an equality
\[
\dbar L_K = \int_{\partial K} \cL,
\]
for any compact submanifold $K$ of~$\CC$.
\end{lemma}

We now want to promote this relationship of classical observables to one between quantum observables.
We do this in two steps.
First, recall that the smeared classical observables are quasi-isomorphic to the (distributional) classical observables,
by the Atiyah-Bott lemma. (See Appendix D of \cite{CG1} as well as Section 2, Chapter 4.)
Hence we replace $L_K$ by a smeared observable $\widetilde{L}_K$ that is cohomologous, and likewise for any classical observable.
Notationally, we will leave this replacement implicit and write simply $L_K$.
Second, every (smeared) classical observable $F$ can be promoted to a quantum observable $\frak{q}(F)$ 
by the cochain isomorphism $\frak{q}$, as discussed in Section~\ref{sec quant map}.

As we want to identify elements of the vertex algebra from observables,
we restrict our attention to the following situation.
Fix radii $0 < s < S < r <R$ and
consider the inclusion 
\[
A_{S<r}(0) \sqcup D_s(0) \hookrightarrow D_R(0)
\]
of an annulus and a small disk into a big disk.
All are centered at the origin. We will consider the factorization product 
\be\label{disk ann disk}
\Obs^{\q}_n(A_{S<r}(0)) \tensor \Obs^{\q}_n (D_s(0)) \to \Obs^{\q}_n(D_R(0)) .
\ee
At the level of vertex algebras, this map corresponds to the action of ``fields'' (which live on annulus and thus depend on $z$ and $z^{-1}$) on ``states'' (which live on a disk and hence in the state space).

\begin{dfn}
Let $L$ be a constant-coefficient local functional that is $\beta$-linear or $\beta$-free.
Its {\em disk observable} $L_{disk}$ is $L_{\{|z| \leq r\}}$. For the circle $S^1_r$ of radius $r$, its {\em circle observable} $L_{circ}$ is $\dbar L_{disk}$.
\end{dfn}

The circle observable $L_{circ}$ is an element of $\Obs^{\q}_{n}(A)$ where $A$ is any annulus containing the circle of radius $r$. Note that a circle observable should be identifiable with a Fourier mode of some field for the vertex~algebra.

\begin{lemma}
\label{der vs star}
Let $L$ be a constant-coefficient local functional that is $\beta$-linear or $\beta$-free.
For any cocycle $F \in \Obs^\q_n(D_s(0))$, we have
\[
\hbar \left[ \{\frak{q}(L) , F\} \right] = \left[\frak{q}(L_{circ}) \cdot F\right]
\]
at the level of cohomology, where $\cdot$ denotes the factorization product in (\ref{disk ann disk}), and where $L_{circ}$ is the circle observable supported on some circle contained in the annulus $A_{S<r}(0)$.
\end{lemma}

\begin{proof}
We compute
\begin{align*}
\d^\q(\frak{q}(L_{disk}) F) &= \d^\q(\frak{q}(L_{disk})) F \pm \frak{q}(L_{disk}) \d^\q F+ \hbar \{\frak{q}(L_{disk}),F\} \\
&=\frak{q}(L_{circ})F +\hbar \{\frak{q}(L_{disk}),F\},
\end{align*}
since $\frak{q}$ intertwines the classical differential $\dbar $ and quantum differential $\d^\q$ and
since the support of $F$ is contained in the disk $|z| < r$, 
on which $L$ and $L_{disk}$ are indistinguishable.

At the level of cohomology, we thus obtain
\[
[\frak{q}(L_{circ}) \cdot F] =\hbar \left[ \{\frak{q}(L) , F\} \right] 
\]
as claimed.
\end{proof}

\begin{ex} Consider the local functional
\[
L(\gamma,\beta) = \frac{1}{2\pi i}\int \gamma \wedge \beta. 
\]
Note that
\[
L_{circ} (\gamma,\beta) = \frac{1}{2\pi i}\int_{|z|=1} \gamma \wedge \beta,
\]
is precisely the functional $F$ from Section \ref{circ ex}.
We showed there that $L_{circ}$ is cohomologous to the functional
\[
H(\gamma,\beta) = \frac{1}{(2\pi i)^2} \int_{|z| = r} \int_{|w| = R} \frac{\gamma(z) \d z \wedge \beta(w)}{z-w},
\]
with $R > r$.
This functional $H$ is manifestly the zeroth Fourier mode of $c_0 b_{-1}$,
by the discussion in Section~\ref{identifications}.

The zeroth Fourier mode of $c_0 b_{-1}$ acts like a number operator or Euler vector field, in the sense that 
\[
(c_0 b_{-1})_{(0)} f = p f 
\]
when $f$ is a homogeneous polynomial of degree $p$ in the variables $c_{m},b_{l}$, with $m \leq 0$ and~$l < 0$.

On the other hand, note that $\frak{q}(L) = L + C$, where $C$ is a constant, since $L$ is quadratic and hence only admits at most one nontrivial contraction with the propagator $P$. Hence, the derivation $\{\frak{q}(L), -\}$ agrees with the derivation $\{L, -\}$.
Direct computation of this derivation shows that it also counts the number of incoming $\gamma$ and $\beta$ legs into any observable; it is the number operator. At the level of the vertex algebra, it thus recovers the zeroth Fourier mode of $c_0 b_{-1}$,
as claimed by the lemma.
\end{ex}

Let us build on this example to get a general statement.
As a matter of notation, if $A_{S<r}(0) \hookrightarrow D_R(0)$ denotes the inclusion of the annulus inside of the disk, 
denote the resulting structure map of the factorization algebra by $\iota : \Obs^{\q}_n(A_{S<r}(0)) \to \Obs^{\q}_n(D_R(0))$. 
In the language of vertex algebras, this map sends a field $A$ to the state~$A |0\rangle$.

For an arbitrary constant-coefficient local functional, 
we then have the following relation between the vertex algebra and in the factorization algebra.

\begin{lemma}\label{zero fourier mode} 
Let $L$ be a local functional that is $\beta$-linear or $\beta$-free, 
and let $\cdot$ be the factorization product (\ref{disk ann disk}). 
For each disk observable $O \in \Obs^{\q}_n(D_s(0))^{(k)}$, 
\ben
[\frak{q}(L_{circ}) \cdot O] = \hbar [\iota(\frak{q}(L_{circ}))]_{(0)} [O] .
\een
That is, $L$ determines a vertex algebra derivation that is {\em inner}.
\end{lemma}

\begin{proof}
For notational convenience, we assume $n = 1$. 
Moreover, it suffices to assume that $c$ is a linear observable, since the operations are derivations and one can thus apply the Leibniz rule.

Let $L$ be a functional of the form $\int \gamma^{\wedge p} \wedge \beta$.
It follows that the circular observable is given by 
\[
L_{circ} = \int_{|z| = 1} \gamma^{\wedge p} \wedge \beta.
\] 
In the vertex algebra $\hCDO_1$, the element $(c_{0})^p b_{-1}$ corresponds to the cohomology class of~$L_{circ}$. 

Recall that a linear observable on the disk is a linear combination of observables of the form
\begin{align*}
O_{c_{-m}}\gamma,\beta) &= \frac{m!}{2\pi i} \int_{|z| = } \frac{\gamma(z)}{z^{m+1}} \d z \\
O_{b_{-l}}(\gamma,\beta)  &= \frac{(l-1)!}{2\pi i} \int_{|z| = } \frac{\beta(z)}{z^l}
\end{align*}
where $m \geq 0$, $l > 0$. 
We will compute the cohomology class of $L_{circ} \cdot O_{c_{-m}}$ in $\Obs^\q(D_{R}(0))$ and demonstrate the claim explicitly  in this case. We leave the case of $O_{b_{-l}}$ for the reader, as it follows a parallel treatment. 

Note that $\frak{q}(L_{circ}) = L_{circ}$ because both $O_{c_{-m}}$ and $O_{b_{-l}}$ involve a factor of $\d z$. Also, $\frak{q}(O_{c_{-m}}) = O_{c_{-m}}$. Moreover, we have
\ben
\partial_P \left(L_{circ} \cdot O_{c_{-m}} \right) (\gamma,\beta) = \frac{m!}{2\pi i}  \int_{|z| = } \frac{\gamma(z)^{\wedge p}}{z^{m+1}}  .
\een
Thus $(L_{circ} \cdot O_{c_{-m}})(\gamma) =  \hbar \frac{m!}{2\pi i}  \int_{|z| = } \frac{\gamma(z)^{\wedge p}}{z^{m+1}}$. 
On the other hand, the zeroth Fourier mode of $c_0^pb_{-1}$ applied to $c_{-m}$ is computed as
\ben
(c_0^p b_{-1})_{(0)} (c_{-m}) = (c_0^p b_{-1})_{(0)} (T^m c_{0}) = T^m ((c_0^p b_{-1})_{(0)} c_0) = T^m(c_0^p),
\een
which is precisely the cohomology class of the observable above. 

The proof for local functionals that are $\beta$-free is completely analogous. 
\end{proof}
 
\subsection{The action of $\TVect$}

The preceding discussion was abstract but there are two local functionals that play an important role for us:
the local functionals $I^{{\rm W}}$ and $J$ produced by equivariant BV quantization.
As discussed in Section \ref{extendedtheory} in Part II, the local functionals encode how $\TVect$ acts on 
the rank $n$ formal $\beta\gamma$ system.
Specifically, we showed that this equivariant quantization equipped the the factorization algebra
$\Obs^{\q}_n$ with the structure of a semi-strict $(\TVect,\GL_n)$-module. 
Moreover, we showed that this semi-strict action induces a strict action of $(\TVect,\GL_n)$ on the cohomology $H^*\Obs^{\q}_n$.

Our goal now is to use the tools we just introduced to describe the strict action of $\TVect$ on $\hCDO_n$
determined by these local functionals. 
Recall that there exists a Lie algebra map $\rho : \TVect \to \Der_{{\rm VA}}(\hCDO_n)$,
by Theorem \ref{MSV1} from Section \ref{sec hc cdo}.
Explicitly, viewing a pair $(X, \omega) \in \Vect \times \hOmega^2_{n,cl}$ as an element of $\TVect$
as in Section \ref{sec hc cdo}, 
we have
\[
\rho(f(t) \partial_j,0) = (f(c) b_{-1}^j)_{(0)}
\]
and
\[
\rho(0, \d(f(t) \d t_j)) = \left(f(c) T (c_0^{j})\right)_{(0)}.
\]
(On a formal disk, every closed 2-form is exact, so it suffices to give the formula in terms of a 1-form $f(t) \d t_j$.)
These vertex algebra derivations are manifestly \emph{inner}, i.e., come from elements of the state space of~$\hCDO_n$.

\begin{lemma}
For $X \in \Vect$, the local functional $I_X^{\rm W}$ determines 
a derivation $\{\frak{q}(I_X^{\rm W}),-\}$ of the factorization algebra $\Obs^\q_n$ 
whose associated vertex algebra derivation is~$\rho(X,0)$.
\end{lemma}

\begin{proof}
Every formal vector field is a linear combination of vector fields with monomial coefficients,
so we simply consider  $X = t^{m_1}_1 \cdots t^{m_n}_n \partial_j$.
The associated local functional is
\ben
I^{\rm W}_X (\gamma, \beta) =\int_\CC\gamma_1^{m_1} \wedge \cdots \wedge \gamma_n^{m_n} \wedge \beta_j,
\een 
which is constant-coefficient holomorphic and $\beta$-linear.
By Lemma \ref{der vs star}, we know that we can understand the derivation $\{\frak{q}(I^{\rm W}_X),-\}$ 
through the factorization product with $(I^{\rm W}_X)_{circ}$, 
which should correspond to the Fourier mode of some element of~$\hCDO_n$.

By Lemma \ref{zero fourier mode} we find that for $X = t^{m_1}_1 \cdots t^{m_n}_n \partial_j$,
the factorization product by $(I^{\rm W}_X)_{circ}$ corresponds, at the level of the vertex algebra, to the zeroth Fourier mode of~$(c^1_0)^{m_1} \cdots (c^n_0)^{m_n} b^j_{-1}$, as desired. Thus, we recover precisely the formula for~$\rho$.
\end{proof}

Likewise, we have the following.

\begin{lemma}
For $\omega \in \hOmega^2_{n,cl}$, the local functional $J_\omega$ corresponds to the vertex algebra derivation $\rho(0,\omega)$.
\end{lemma}

\begin{proof}
Every closed 2-form $\omega$ on the formal disk is the exterior derivative $\d \theta$ of a 1-form $\theta$. 
Moreover, every 1-form is a linear combination of 1-forms with monomial coefficients,
so we simply consider  $\theta = t^{m_1}_1 \cdots t^{m_n}_n \d t_j$.
The associated local functional is
\ben
J_{\d \theta} (\gamma, \beta) = \int_\CC\gamma_1^{m_1} \wedge \cdots\wedge \gamma_n^{m_n} \wedge \partial_z \gamma_j \,\d z,
\een 
which is constant-coefficient holomorphic and $\beta$-free.
The derivation $\{\frak{q}(J_{\d \theta}),-\}$ corresponds to the factorization product with the circle observable
\[
(J_{\d \theta})_{circ} (\gamma, \beta) =\int_{|z|=1}\gamma_1^{m_1} \wedge \cdots\wedge \gamma_n^{m_n} \wedge \partial_z \gamma_j \,\d z.
\]
Moreover, this circle observable corresponds to the zeroth Fourier mode of $(c^1_0)^{m_1} \cdots (c^n_0)^{m_n} T(c^j_0)$, by Lemma \ref{zero fourier mode}, as desired.
\end{proof}

\section{The main result}

In light of our arguments in the preceding sections, we obtain the following.

\begin{thm} 
The isomorphism of $\ZZ_{\geq 0}$-graded vertex algebras
\ben
\Phi : \hCDO_n \xto{\cong} \Vert(\Obs^{\q}_n) 
\een
is equivariant with respect to the actions of $(\TVect, \GL_n)$. 
Moreover, it is compatible with the $\hO_n$-module structure. 
\end{thm}

\begin{proof}
We proved the equivariance assertion in the preceding sections. 
Thus it remains to discuss the $\hO_n$-module structure.
This aspect, however, is the focus of Section \ref{sec concrete},
where we describe how observables on a disk decompose according to conformal dimension
and then identify each subspace of fixed conformal dimension with some type of formal tensor fields.
At the level of cohomology --- which provides the decomposition for $\Vert(\Obs^{\q}_n)$ --- these match with~$\hCDO_n$.
\end{proof}

An immediate corollary, via Gelfand-Kazhdan descent and its variants, is our main result.

\begin{cor} 
Let $X$ be a complex $n$-manifold together with a trivialization $\alpha$ of $\ch_2(T_X) \in H^2(X ; \Omega^{2,hol}_{cl})$. Then the factorization algebra $\Obs^{\q}_{X,\alpha}$ obtained by Gelfand-Kazhdan descent 
determines a sheaf of vertex algebras $\Vert(\Obs^{\q}_{X,\alpha})$ on $X$. 
Moreover, there is an isomorphism of sheaves of vertex algebras on $X$
\ben
\Phi : \CDO_{X,\alpha} \xto{\cong} \Vert(\Obs^{\q}_{X,\alpha})
\een
that is natural in the choice of trivialization~$\alpha$.
\end{cor}

The naturality in the choice of trivialization can be phrased in a compelling way.
Recall from Section \ref{sec obsdef} that there is an obstruction-deformation complex for the curved $\beta\gamma$ system on $X$, which is a sheaf of dg vector spaces encoding important information about this BV theory.
For instance, a degree one cocycle encodes a first-order deformation of the classical action that satisfies the classical master equation and thus defines a well-posed classical BV theory.
This complex is obtained by Gelfand-Kazhdan descent, and it involves only local functionals that are invariant for the action of $\CC^\times \times {\rm Aff}(\CC)$ by scaling (of the $\beta$ field) and affine transformations.

In particular, the obstruction to quantization is a degree two cocycle in this cochain complex, 
which is identified with $\ch_2(T_X)$ by Lemma \ref{obsprop}.
Corollary \ref{gerbe of obsdef} tells us that this sheaf of dg vector spaces is quasi-isomorphic to the sheaf $\Omega^{2,hol}_{cl}$ on $X$. Hence we deduce the following.

\begin{cor}
The map $\Phi$ provides an isomorphism of gerbes from the gerbe of BV quantizations of the curved $\beta\gamma$ system (constructed via descent and invariant under $\CC^\times \times {\rm Aff}(\CC)$) to the gerbe of CDOs.
\end{cor}

\subsection{Remark on conformal structure}

With the identification of chiral differential operators with the observables of the $\beta\gamma$ system, 
our analysis in Part II immediately implies an observation about the conformal symmetry of this sheaf of vertex algebras. 

In Section \ref{sec conformal obs} we showed that after fixing a trivialization $\alpha$ of $\ch_2(T_X)$, 
there is a map of sheaves on~$X$ 
\ben
\Psi^\q : \ul{\sVir}_{c=2n} \to \Obs^\q_{X, \alpha}
\een 
of factorization algebras on $\CC$, {\em provided} that $c_1(T_X) = 0$. (In fact, we have such a map for every trivialization of~$c_1(T_X)$). 

The factorization algebra $\sVir_{c}$ is amenably holomorphic, and it is shown in \cite{bw_vir} that its associated vertex algebra $\Vert(\sVir_{c})$ is isomorphic to the Virasoro vertex algebra ${\rm Vir}_c$ of central charge $c$. 
By the functoriality of the functor $\Vert$, 
we obtain the following immediate corollary of the above analysis, which implies the aforementioned Proposition \ref{prop conformal cdo} from Part~I. 

\begin{cor} 
Let $\alpha$ be trivialization of $\ch_2(T_X)$ and let $\CDO_{X,\alpha}$ be the associated sheaf of CDOs. 
Then for each trivialization $\beta$ of $c_1(T_X)$, 
the map of holomorphic factorization algebras $\Psi^\q_\beta : \ul{\sVir}_{c=2n} \to \Obs^\q_{X,\alpha}$ determines a map of sheaves of vertex algebras $\Vert(\Psi^\q_\beta) : \ul{\rm Vir}_{c=2n} \to \CDO_{X,\alpha}$.
\end{cor}

\section{Discussion of some physics literature}

Our goal in this section is to relate our work to the perspectives offered by Witten and Nekrasov on the curved $\beta\gamma$ system.
Both \cite{WittenCDO} and \cite{Nek} undertake a similar analysis, but we will focus on Witten's.
The format of our comparison is to remind the reader about general aspects of $\sigma$-models,
to explain how Witten identifies the anomalies to quantization and how his method relates to ours,
and to indicate how Witten determines the patching rules for the chiral algebras and how this approach relates to ours.

\subsection{General comments about nonlinear $\sigma$-models}

\def\Maps{{\rm Maps}}

We begin by sketching a general perspective that informs the problem.

\begin{rmk}
This perspective assumes that the path integral exists and exhibits behavior analogous to finite-dimensional integrals.
In a sense, we run the argument sketched here {\it backwards} to construct the putative path integral measure.
\end{rmk}

Let $\Sigma$ denote a source manifold and $X$ a target manifold.
A nonlinear $\sigma$-model has, as its space of fields, the infinite-dimensional manifold $\Maps(\Sigma,X)$.
The equations of motion for the theory cut out a submanifold $Sol$ of this mapping space as the space of solutions,
and typically a component of this space of solutions is given by a copy of $X$ viewed as the constant maps from $\Sigma$ to $X$.
We will call this the {\em perturbative} sector.

In trying to compute the path integral, one expects that for $\hbar$ very small, 
the putative measure should be concentrated very close to $Sol$ inside $\Maps(\Sigma,X)$.
One might then try to approximate the path integral by simply integrating over a small tubular neighborhood around $Sol$.
The perturbative contribution would then be the integral over a small tubular neighborhood around $X$ inside $\Maps(\Sigma,X)$.
To organize the computation of this perturbative contribution, one can identify a tubular neighborhood with the normal bundle to $X$ inside $\Maps(\Sigma,X)$.
Hence, one obtains an infinite-dimensional vector bundle over $X$ whose fiber over $x \in X$ is 
\[
N_x \Maps(\Sigma,X) \cong T_x \Maps(\Sigma,X)/T_x X \cong \Maps(\Sigma, T_x X)/T_x X.
\]
One then computes the integral over the tubular neighborhood in two stages: 
first, fiberwise integration over the normal bundle, and then integration over $X$.
This fiberwise integral can be approached with Feynman diagrammatics,
with the base $X$ playing the role of a ``background field.''
In this sense, the perturbative sector is {\em local on the target~$X$}

A better approximation to the full path integral would involve the other components of $Sol$. 
They provide the ``instanton corrections'' to the perturbative computation.
As a nonconstant solution is not concentrated at a single point in $X$ --- by definition --- 
these corrections are not local on $X$ and require different techniques.

\subsection{Anomalies and obstructions}

Fix some method for perturbative computations.
At each point $x$ of $X$, we apply our method to integrate over the normal bundle $N_x$.
This integral ought to take values in a one-dimensional vector space, and 
hence the full fiberwise integral provides a section in a line bundle over $X$.
{\em A priori} we do not know which line bundle it is, 
since the perturbative constructions are done locally on $X$ and then patched together.

As Witten notes in Section 2.3 of \cite{WittenCDO}, the Chern class of this line bundle is a discrete invariant and 
hence should not depend on continuous parameters, 
such as the coupling constants of the fiberwise perturbative field theories.
Thus we can compute it by fiberwise quantizing the family of free theories over $X$,
scaling the interactions to zero.
In other words, one simply keeps the Hessian of the action functional at each point $x \in X \subset \Maps(\Sigma,X)$.
The free theory at each $x$ corresponds to some elliptic complex on $\Sigma$. 
Now, it is standard to identify the one-dimensional vector space at $x \in X$ with the determinant line of the cohomology of this elliptic complex.
(This identification can be recovered as a {\em consequence} of BV quantization, as shown in \cite{GwHaug}.)
Hence, one can use a families index theorem to compute the Chern class of the determinant line bundle.

In the case of the $\beta\gamma$ system, 
this amounts to considering the trivial fiber bundle $\pi: \Sigma \times X \to X$
and letting the elliptic complex at a point $x \in X$ be $\Omega^{0,*}(\Sigma) \otimes T_x X$.
In other words, we are considering $\cF = \cO_\Sigma \otimes \cT_X$ as a sheaf on $\Sigma \times X$.
We wish to understand the pushforward $\pi_* \cF$ on $X$ or, more accurately, the determinant line of its derived pushforward.

\def\Td{{\rm Td}}

We assume now that $\Sigma$ is closed.
The first Chern class of the determinant line agrees with the first Chern class of the derived pushforward.
The Grothendieck-Riemann-Roch theorem then implies that the Chern character of the derived pushforward 
is given by 
\begin{align*}
\pi_*( \ch(\cF) \Td(\cT_\pi)) &= \pi_*( (1 + c_1(\cF) + \frac{1}{2}(c_1(\cF)^2 - 2c_2(\cF)) + \cdots)(1 +\frac{1}{2}c_1(T_\Sigma)))\\
&= \pi_*( (1 + \pi^* c_1(\cT_X) + \pi^* \ch_2(\cT_X) + \cdots)(1 +\frac{1}{2}c_1(T_\Sigma)))
\end{align*}
and the first Chern class is the component of cohomological degree 2, namely
\[
(1-g) c_1(\cT_X) + \ch_2(\cT_X) 
\]
where $g$ denotes the genus of $\Sigma$.
In short, one finds that the determinant line is trivial if and only if 
both $\ch_2(\cT_X) = 0$ and either $c_1(\cT_X) = 0$ or $\Sigma$ is genus one.

These results correspond to ours,
although our approach is, on its face, rather different.
We choose to work with the formal $n$-disk as target 
and then apply Gelfand-Kazhdan descent to obtain (the perturbative sector of) the theory with any complex $n$-manifold $X$ as target.
Our obstruction cocycle (or anomaly) thus lives in Gelfand-Fuks cohomology
and maps to de Rham cohomology of some $X$ by descent.
As we have seen in \ref{sec equiv bv}, the obstruction to BV quantization descends to $\ch_2(\cT_X)$ when the source is $\CC$ 
and we require equivariance with respect to
\begin{itemize}
\item translation and dilation on the source (which ensures we can descend to genus one curves) and
\item holomorphic diffeomorphisms of the formal $n$-disk as target.
\end{itemize}
It is a consequence of the calculations of Section \ref{sec conformal anomaly} combined with Gelfand-Kazhdan descent that requiring equivariance under {\em all} holomorphic diffeomorphisms of a source disk requires $c_1(\cT_X) = 0$.
Such equivariance ensures one can descend to higher genus curves by considering them as quotients of the disk by Fuchsian groups.

\begin{rmk}
On the other hand, Witten's global argument using Grothendieck-Riemann-Roch applies in the BV context as well,
because BV quantization of the free $\beta\gamma$ system 
on a Riemann surface $\Sigma$ but twisted by $\cT_X$ recovers the same determinant line, 
up to some cohomological shift.
A benefit of our approach here is that we actually construct the BV quantization for the full $\beta\gamma$ system
rather than merely identifying the obstruction cohomologically.
\end{rmk}

As a further point of comparison, note that the obstruction-deformation complex we compute in Section \ref{sec eq def cplx} recovers precisely the same information that Witten and Nekrasov find.
For instance, they find that first-order deformations of the theory are given by $H^1(X, \Omega^2_{cl})$.
(See Section 2.2 of \cite{WittenCDO} or Section 2.6.3 of \cite{Nek}, although Nekrasov keeps track of deformations of complex structure of the target too.)
Similarly, one can see that the local symmetries of the theory are $H^0(X, \Omega^2_{cl})$, as Nekrasov notes in Section 2.6.6.
In other words, the BV formalism provides a systematic mechanism for answering the questions that Witten and Nekrasov address.

\subsection{Chiral algebras and observables}

In Section 3 on \cite{WittenCDO}, Witten explains how one can recover the sheaf of chiral differential operators by physical arguments.
His approach might be summarized as follows:
\begin{enumerate}
\item since the perturbative theory is local on the target $X$, we fix a good cover $\{U_i\}$ of $X$ and try to patch the quantizations;
\item a coordinatization $\phi_i: U_i \hookrightarrow \CC^n$ allows one to view the field theory as a restriction of the $\beta\gamma$ system with target $\CC^n$ to the open $\phi_i(U_i)$; 
\item one then constructs the chiral algebra of operators for the open $\phi_i(U_i)$ as target by restricting the chiral algebra for the free $\beta\gamma$ system of rank $n$;
\item one tries to patch the chiral algebras on overlaps $U_i \cap U_j$.
\end{enumerate}
The first two steps are built into this perturbative approach to general $\sigma$-models.
The third step depends on two things: 
first, knowing the chiral algebra of the free $\beta\gamma$ system 
(which is a standard computation in physics and which we formalized in Part I), 
and second, knowing the chiral algebra can be localized to smaller opens.
(We remark that {\em chiral} algebra here is synonymous with {\em vertex} algebra in the mathematics literature,
although physicists often (though not here) presume that a chiral algebra is invariant under holomorphic changes of coordinates 
and hence what a mathematician might call a vertex operator algebra.)
This second result is a computation done in \cite{MSV, GMS},
and it requires one to show that the OPE for the free $\beta\gamma$ system can be localized from polynomials in the coordinates on the target $\CC^n$ to holomorphic functions in those coordinates.

The final step is a bit involved, and Witten explains it in Section 3.4 of \cite{WittenCDO}.
He wants to patch the chiral algebras for small opens in $\CC^n$,
so he needs to identify the automorphism group of the chiral algebra.
In practice, he instead computes the Lie algebra of this automorphism group 
or, more accurately, the infinitesimal automorphisms arising from the chiral algebra itself.
Witten wants to find those elements of the chiral algebra whose zeroth Fourier mode 
acts on the chiral algebra as a derivation that preserves conformal dimension.
In his terminology, such an element is a ``dimension one current'' and 
its zeroth Fourier mode is called its ``charge,'' 
which is the integral of the current along a loop around the origin in the source manifold.
Witten uses $\fg$ to denote the Lie algebra given by integrals of dimension one currents modulo total derivatives.

Witten provides two natural types of symmetries.
A holomorphic vector field $V = V^i \partial_i$ on the target determines the current $J_V = -V^i \beta_i$,
viewed as an element of the chiral algebra.
Similarly, a holomorphic 1-form $B = B_i \d z^i$ determines a current $J_B = B_i \partial \gamma^i$.
The charge of a 1-form $B$ vanishes if and only if it is exact (i.e., $B = \partial H$),
so that the collection of such charges is isomorphic to closed 2-forms (as a vector space).
The charges of distinct vector field are, by contrast, distinct.
Let $\frak{v}$ denote the Lie algebra formed by the charges for vector fields $V$ and 
let $\frak{c}$ denote the Lie algebra formed by the charges for one-forms $B$.
Together they span $\fg$, according to Witten.

A direct computation with the charges (or using OPE with these currents) shows that there is an exact sequence of Lie algebras
\[
0 \to \frak{c} \to \fg \to \frak{v} \to 0,
\]
and this Lie algebra corresponds to the extension $\widetilde{W}_n$ that we construct.

In our setting, a current can be defined as a local functional $I$ such that $\{I,-\}$ is a cocycle in derivations of the observables.
As discussed in Section \ref{sec loc sym}, such a functional $I$ determines an element $I_{disk}$ in the observables on the disk and hence an element of the vertex algebra.
The associated charge is the element $I_{circ}$, which is an observable on the annulus and hence corresponds to the zeroth Fourier mode of the element in the vertex algebra.

Witten chooses to find the infinitesimal symmetries of the $\beta\gamma$ system 
by doing explicit computations in the chiral algebra after obtaining it from the free theory.
By contrast, we use the BV formalism to determine how to lift the classical symmetries to quantum symmetries
(at the cost of a Lie algebra extension)
and then extract the chiral algebra statements.
In a sense, we do path integral manipulations to recover chiral algebra, 
and Witten follows the reverse logic.
The results naturally agree.

\begin{rmk}
Recent work  \cite{LiQME} of Si Li provides another useful perspective on this relationship. 
He considers free holomorphic BV theories on the complex line,
and he produces an identification between the obstruction-deformation complex of the BV theory and the mode Lie algebra of its vertex algebra.
His result, extended to our equivariant context, then recovers Witten's computation. 
See Remark~\ref{rmk on si's work} for an extended discussion of this point.
\end{rmk}

After determining the appropriate symmetries of the chiral algebra, 
Witten tries to lift the patching of coordinates on the good cover to patching of the chiral algebras.
The coordinate patching can be seen as living in the Lie algebra $\frak{v}$ (or rather its assciated group),
so the challenge is to lift to $\fg$, which involves choices.
Any choice determines a \v{C}ech 2-cocycle with values in $\Omega^2_{cl}$,
and if the cocycle vanishes,
the choices determine a patching of the chiral algebras.
The work of \cite{GMS} showed that this cocycle is indeed $\ch_2(T_X)$.
Our method provides another perspective.

